%---------------------------------------%
% Packages arranged by : Tsz Timmy Chan %
%                 Date : May 26th, 2019 % 
%---------------------------------------%

\documentclass{TC}
\usepackage{TCcommon}

\title{TITLE HERE}	% Work Title Here.
\author{Tsz Timmy Chan}	% YOUR NAME HERE 

\usepackage[notes]{TCheader}
\usepackage{TCexamtitle}

\usepackage{setspace}
\linespread{1.5}

%\renewcommand{\benediction}{" " - }
%\renewcommand{\quoteoftheday}{" " \\ - }

\begin{document}

Qualitative Research Traditions:
\begin{itemize}
\item Narrative Inquiry: Biography, autobiography, life history and oral history. Mainly uses interviews and primary documents. Looking for epiphanies, and examines context. \emph{Outcome: Detailed picture of an individual's life.} 
\item Phenomenology: "Pre-sociology" often used by philosophy. Long interviews with $\approx 10$ people. Examines statements, meanings, meaning themes, general descriptions of the experience. \emph{Outcome: "Essence" of the experience.}
\item Grounded Theory: produce a theory/model (Very structured) Interview with 20-30 people. Uses the techniques: Open coding, Axial coding, Selective coding, Conditional matrix. \textit{Outcome: theory \& theoretical model}
\item Ethnography: comes from anthropology - describing a culture and social group. Examines observations, interviews and other artifacts and often immersion over a long time. Analysis done using description and interpretation. \emph{Outcome: description of the culture behavior of a group or an individual.}
\item Case Study: In-depth study of a particular "case". Multiple sources - documents, archival records, interviews, observations, physical artifacts. Analysis: Description, themes, assertions.
\end{itemize}

Qualitative Methods:
\begin{itemize}
\item Document and Archival Analysis: document = data.
\item Observation: systematically taking notes \& recording in naturalistic setting.  
\item Interview: interaction based, one-on-one.
\item Focus Groups: group interviews. Used when one-on-one is limiting. 
\end{itemize}


Qualitative Sampling Strategies (instead of random...):
\begin{itemize}

\item Maximal variation sampling (every category represented)
\item Extreme case sampling (get outlier; extreme characteristics) 
\item Theory/concept sampling (fits operational definition)
\item Homogeneous sampling (something in common)
\item Critical Sampling (challenging a phenomenon)
\item Opportunistic/emergent sampling (due to unfolding events) \textit{This is different from Convenient sampling!}
\item Snowball Sampling (ask participants to refer others)
\end{itemize}


$$(\textbf{Trustworthiness, Rigor}) \simeq (\textbf{Reliability, Validity)}$$

Trustworthiness:
\begin{enumerate}
\item Credibility
\item Transfer-ability
\item Dependability
\item Confirm-ability 
\end{enumerate}

TRIANGULATION \& MEMBER CHECKING are gold standards, because they theoretically should increase the 



\end{document}

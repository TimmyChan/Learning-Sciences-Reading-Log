
%---------------------------------------%
% Packages arranged by : Tsz Timmy Chan %
%                 Date : May 26th, 2019 % 
%---------------------------------------%

\documentclass{TC}
\usepackage{TCcommon}

\title{TITLE HERE}	% Work Title Here.
\author{Tsz Timmy Chan}	% YOUR NAME HERE 

\usepackage[notes]{TCheader}
\usepackage{TCexamtitle}

\usepackage{setspace}
\linespread{1.5}

%\renewcommand{\benediction}{" " - }
%\renewcommand{\quoteoftheday}{" " \\ - }

\begin{document}


Discussed what makes an expert, how to become an expert, how to assist someone else to become an expert, and how to evaluate someone's expertise

Had a discussion on 
Discussed what is \gls{lpp}

Discourse of Gee:

Supplementary Readings, presented by Meerok and Erin:

\begin{itemize}
\item Primary discourse is what you grow up with at home and in families.
\item Secondary discourse arises when one leaves the primary discourse - and secondary first discusses those that involve print, but this also applies to film, TV, etc.
\item Acquisition vs Learning --- 
\end{itemize}  


\gls{ec} How people think about knowledge, what their beliefs are and how it manifests in their way of seeking knowledge. practices are a way to make beliefs about epistimic commitments visible - example, scientist will interrogate data. this way of studying actions reveal the beliefs underneath \parencite{fischer_epistemic_2018}:
\begin{itemize}
\item Features of LS research on \gls{ec}
	\begin{itemize}
	\item emphasizing multidisciplinary research
	\item broadening the range of questions
	\item challenging normative assumptions
	\item a focus on practices
	\item the throroughly social nature of \gls{ec},
	\item its situativity. 
	\end{itemize}
\end{itemize}
\begin{definition}[Epistemic practices]
Socially normed activites that people carry out accomplish epistemic aims such as developing evidence, arguments, theories and so on.
\end{definition}

C

\end{document}

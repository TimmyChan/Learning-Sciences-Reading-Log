
%---------------------------------------%
% Packages arranged by : Tsz Timmy Chan %
%                 Date : May 26th, 2019 % 
%---------------------------------------%

\documentclass{TC}
\usepackage{TCcommon}

\title{TITLE HERE}	% Work Title Here.
\author{Tsz Timmy Chan}	% YOUR NAME HERE 

\usepackage[notes]{TCheader}
\usepackage{TCexamtitle}

\usepackage{setspace}
\linespread{1.5}

%\renewcommand{\benediction}{" " - }
%\renewcommand{\quoteoftheday}{" " \\ - }

\begin{document}
Snapshot of the current state of educational research, and what qualifies as scientific educational research?

"Debate" - can Educational Research be scientific? What is scientific?

Nice quotes:
"All models are wrong, some are useful" --- George Box\\
"Better an approximate answer to the right question than an accurate answer to the wrong question" --- John Tukey
Main Ideas and Takeaways on Feuer et al. ER Paper \parencite{feuer_scientific_2002}: 
	\begin{itemize}
	\item Developmental level of Education Research and its reputation - is this caused by funding, or the age of the field (immaturity)
	\item Science $\neq$ Methods
	\item Role of Government in Educational Research
	\item Culture relate to Education Research Quality; Self-Regulation
	\item Empiricism
	\item Similarity and difference between Education Research and other disciplines
	\item Multiple players $\implies$ multiple disciplines/bias/techniques
	\item Quantitative versus Qualitative Methods - $\exists$ tensions between folks who comes in with different "epidemiological baggage", and the inherent biases, leading to narrow views of what research is considered rigorous. "Educational research has a long history of struggling to become---or to ward off---science.
	\item Becomes a work on definition of \emph{rigor}.
	\item Educational research became a bit of a political 
	\item Science is often about measurement, and is dependent on the tools used to measure things. This means both physical tools as well as mathematics/statistics. 
	\item Science is not free from culture \& politics.
	\item contentions with the public opinion (inference: perhaps a harsher view Education due to personal experience, etc?)
	\end{itemize}




We also read the first 5 chapters of the NRC Report, and got some current understanding of the landscape of scientific educational research \parencite{national_research_council_introduction_2002, national_research_council_accumulation_2002, national_research_council_guiding_2002, national_research_council_features_2002, national_research_council_designs_2002}.

\begin{enumerate}[CH 1:]
\item Report Intro,
\item Accumulation of Knowledge,
\item Guiding Principles for Scientific Inquiry,
\item Features of Education and Educational Research,
\item Designs for the Conduct of Scientific Research in Education
\end{enumerate}
		\begin{itemize}
		\item Science is never finished, but improves warrants for knowledge over time
		\item Nature of progress is the same across fields = zigzagging paths. Progress is a function of time, money, and public support.
		\item Research-based knowledge in education has accumulated in this way, but Education = slower.
		\item 6 principles of Inquiry:
			\begin{enumerate}
			\item Significant questions that can be investigated empirically,
			\item Link research to relevant theory,
			\item Use methods that permit direct investigation of the question,
			\item Provide a coherent and explicit chain of reasoning,
			\item Replicate and generalize across studies,
			\item Disclose research to encourage professional scrutiny and critique. 
			\end{enumerate}
		\item Features of Education and the implications for inquiry
		\item Shelf life $\implies$ persistence?
		\item Ethics and designs for conducting research
		\item Three types of Education Research questions:
			\begin{enumerate}
			\item What's happening;
			\item Is there a systemic effect (casual effects?)
			\item How does it happen? (What is the underlying mechanism?)
			\end{enumerate}
		\end{itemize}
\begin{itemize}[(??)]
	\item Moderator vs Mediator definition between Treatments and Outcome.
\end{itemize}
\begin{itemize}[(!!)]
	\item Scientific $\subset$ Educational Research, but note that there exists valid scholarship that is $(\text{scientific})^c$.
	\item Consider \emph{who} wrote the literature, the \emph{intent} of the writer and people who commissioned the writing (be it publisher or government entity), and the \emph{audience} of the writer. 

\end{itemize}

\end{document}

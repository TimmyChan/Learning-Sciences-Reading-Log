
%---------------------------------------%
% Packages arranged by : Tsz Timmy Chan %
%                 Date : May 26th, 2019 % 
%---------------------------------------%

\documentclass{TC}
\usepackage{TCcommon}

\title{TITLE HERE}	% Work Title Here.
\author{Tsz Timmy Chan}	% YOUR NAME HERE 

\usepackage[notes]{TCheader}
\usepackage{TCexamtitle}

\usepackage{setspace}
\linespread{1.5}

%\renewcommand{\benediction}{" " - }
%\renewcommand{\quoteoftheday}{" " \\ - }

\begin{document}
Scaffolding \parencite{sawyer_scaffolding_2014, fischer_research_2018} and Technology \parencite{sawyer_knowledge_2014}
\begin{itemize}
\item Work is shared between the learner and some more knowledgeable other or agent
\item Scaffolding enables the performance of a task more complex than the learner could handle alone, and enables learning from that experience.
\item \gls{zpd}: \glsdesc{gls-ZPD}
\item \gls{prolepsis}: \glsdesc{prolepsis}
\item As the learner appropriates this guidance and begin to regulate their own actions as the tutor gradually reduces guidance resulting in \emph{fading} of scaffolding
\item Scaffolding is more than just breaking down a problem into smaller sub-tasks (\textit{decomposition}); scaffolding is using a complex (expert level) task, which motivate developing subskills and requisite knowledge, applying knowledge and skill as they are \underline{constructed}.
\item scaffolding these aspects of work can make tasks tasks more productive for learning:
	\begin{enumerate}
	\item Sense-making: helping learners make sens of problems or data.
	\item Articulation \& reflection: helping learners articulate their thinking as they progress on problems
	\item Managing investigation \& problem-solving processes: helping learners with strategic choices and executing 
	\end{enumerate}
\item Possible benefits of scaffolding according to \parencite{sawyer_scaffolding_2014}:
	\begin{itemize}
	\item Simplifies elements of tasks so they are within reach of learners,
	\item Manage the process so learners can engage in elements of the disciplinary work in real problem contexts,
	\item Offset frustration and risk and maintain interest,
	\item Focus learners' attention on aspects of the problem they may take for granted,
	\item Prompt learners to explain and reflect,
	\item Enable learning by doing in context.
	\end{itemize}
\item According to there are three types of \gls{distributed scaffolding}: 
	\begin{itemize}
	\item \Gls{differentiated scaffolding}
	\item \Gls{redundant scaffolding}
	\item \Gls{synergistic scaffolding}
	\end{itemize}
\end{itemize}

Game based learning %\parencite{}

\end{document}


%---------------------------------------%
% Packages arranged by : Tsz Timmy Chan %
%                 Date : May 26th, 2019 % 
%---------------------------------------%

\documentclass{TC}
\usepackage{TCcommon}

\title{TITLE HERE}	% Work Title Here.
\author{Tsz Timmy Chan}	% YOUR NAME HERE 

\usepackage[notes]{TCheader}
\usepackage{TCexamtitle}

\usepackage{setspace}
\linespread{1.5}

%\renewcommand{\benediction}{" " - }
%\renewcommand{\quoteoftheday}{" " \\ - }

\begin{document}
Different disciplines have different established types of methods.

\begin{itemize}
\item Natural Sciences 
	\begin{itemize}
	\item Usually a quantitative approach
	\end{itemize}
\item Humanities
	\begin{itemize}
	\item Art, art history, music, philosophy, etc.
	\item Contrast with social sciences
	\item Significant overlap between social sciences \& humanities. 
	\item Not necessarily seeking to make claims -
	\item "What something is vs. the meaning we attach to it".	
		\begin{itemize}
		\item How disability has come to mean what it means,
		\item How those meanings come to be,
		\item How have the ways meaning been contested.
		\end{itemize}
	\item Critical \& Analytical
	\item "What does it mean to be human"?
	\item more \emph{philosophical} than social sciences
	\item within disabilities studies, the humanities perspective focuses on intersections with other critical fields with the purpose of social activism
	\item Struggle with the meaning of words---the way they are defined and recognize that words can be in flux and they are contextual.
	\item text can be data, where one can analyze the text itself.
	\item Some methods:
		\begin{itemize}
		\item Content Analysis
		\item \gls{CDA} --- this can be different between Humanities and Social Sciences! 
		\item Media
		\item LIteracy
		\item Art History, Art Production, etc.
		\end{itemize}
	\end{itemize}
\item Social Sciences
\end{itemize}


\textbf{types of research}
\begin{itemize}
\item Cross-sectional research: snapshot
\item Longitudinal research - over a length of time
\item Comparative research - comparison in the data, within group vs between group
\item intervention research - pre-post test design
\item Instrument development: new measure or adapting an existing one.
\item Evalutation research: formative vs summative  
\end{itemize}
\end{document}


%---------------------------------------%
% Packages arranged by : Tsz Timmy Chan %
%                 Date : May 26th, 2019 % 
%---------------------------------------%

\documentclass{TC}
\usepackage{TCcommon}

\title{TITLE HERE}	% Work Title Here.
\author{Tsz Timmy Chan}	% YOUR NAME HERE 

\usepackage[notes]{TCheader}
\usepackage{TCexamtitle}

\usepackage{setspace}
\linespread{1.5}

%\renewcommand{\benediction}{" " - }
%\renewcommand{\quoteoftheday}{" " \\ - }

\begin{document}
Knowledge Produced About Disability vs. Disability Studies

Examined the growth and change in philosophy in methodologies, especially towards transdisciplinary ideals and emancipatory research \parencite{barnes_what_2003, kitchin_researched_2000, goodley_decolonizing_2012,
tregaskis_disability_2005}. 

When interviewing, especially with an \gls{intermediary}, it's important to conduct \gls{member checking}.

$\exists$ many instances of unethical and exploitative research where the researchers objectify those with disabilities and ignore the agency or boundaries of those with disabilities.

\begin{definition}[\Gls{reliability}]
\glsdesc{reliability}
\end{definition}
\begin{itemize}
\item Inter-rater or Inter-Observer: multiple observers to extract data or evaluate information, to make sure that the measure is objective by consensus. ($n \geq 3$ and $n$ should be odd).

\item Test-Retest: Give the test multiple times, it should give same results.

\item Parallel-Form: Same content, but different versions of the same test. 

\item Internal Consistency: Same question at different points to make sure that the response is consistent.  
\end{itemize}

\begin{definition}[\Gls{validity}]
\glsdesc{validity}
\end{definition}

\begin{itemize}
\item Internal Validity: The degree to which the results are associated with the independent variable; make sure that no other factors are affecting the outcomes

\item External Validity: Generalizability of the test across different studies

\item Construct Validity: The degree to which a test is actually measuring what it is intended to

\item Criterion-Related Validity: Is the research fitting pre-determined criterion?

\item Content Validity: Are the tests fitting the purpose of the research?

\item Ecological Validity: Is the research reflecting the world? (The setting chosen)
\end{itemize}


\vspace{-2em}
$$\text{Reliability}\equiv \text{Consistency (how far spread)} \;\; \text{Validity} \equiv  \text{Accuracy (how on target)} $$

\underline{\textit{Bias}}
Goal is to minimize bias (increase reliability/validity), and to do so one should learn to spot such bias in one's own and others' research.

\begin{itemize}
\item Sampling Bias: issues with choosing a sample that is not representative of the target population.

\item Selection Bias: Not randomly assigning folks between control/test groups. (Self-selection bias, where individuals choose to participate.)
\item Response Bias: People may give inaccurate answers because they are trying to give answers that they think the researcher wants to hear. (Yea-saying) Sometimes the researcher push until they get the answer they're looking for, or sometimes limit the research questions.
\item Hawthorne Effect: Some times the interaction with researchers can change the subjects' lifestyle; like if they have to keep track of their food/exercise, they may be a bit more "accountable". This can be moderated by control groups.  
\item Performance Bias: Researcher / participants are going to be changing their behavior based on their preconceived associations. Can be mediated by "blind" or "double blind" trials.	
\item Measurement Bias: people measuring or assessing the outcomes in a study should not know which individuals are in a test group or a control group.
\end{itemize}

Disability studies should benefit the people with disabilities and focus on removal of barriers in society. Power should be given to people with disability in creation of knowledge in disability studies. Research should include multitude of methods, so that we can give a more complete and complex view. 

Internalized ableism can be an influence on whether people identify as a disabled person. Not everyone who is disabled would identify as such, and this can introduce selection bias. 

Researchers have a platform, and thus a responsibility to collaborate with people with disabilities in giving voice to a marginalized community.
$$ \text{Inclusive research} = \text{Emancipatory} \cup \text{participatory research}$$

co-production is the outcome of1 participatory research.

Emancipatory research has the goal of empowerment, though transformation of the social dynamics of research; use the privilege of being a researcher to help remove barriers. 

Intersectional feminism and Queer/Crip approaches 

Participatory research: Empirical research that involves the participants in the decision making process. 

Action research: there should be an actionable outcome. 

Participatory Action Research (PAR): participation at each step of the research. 
Community Based Participatory Research (CBPR): Ecologically valid!

$CBPR \subset PAR$.


\end{document}

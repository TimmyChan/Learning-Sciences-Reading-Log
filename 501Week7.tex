
%---------------------------------------%
% Packages arranged by : Tsz Timmy Chan %
%                 Date : May 26th, 2019 % 
%---------------------------------------%

\documentclass{TC}
\usepackage{TCcommon}

\title{TITLE HERE}	% Work Title Here.
\author{Tsz Timmy Chan}	% YOUR NAME HERE 

\usepackage[notes]{TCheader}
\usepackage{TCexamtitle}

\usepackage{setspace}
\linespread{1.5}

%\renewcommand{\benediction}{" " - }
%\renewcommand{\quoteoftheday}{" " \\ - }

\begin{document}
Notes from Goldman Article:
\begin{itemize}
\item Discourse analysis of written text provides a method for systematically describing texts that students read as well as those they write. 

	
\begin{definition}[Readability]
	Reading difficulty of a passage, where traditional formulas fail to consider the familiarity of the concepts in the passage.
\end{definition}
\begin{definition}[Proposition (literacy research analysis)]
	Proposition is a theoretical unit of analysis that corresponds roughly to the meaning of a clause.  Propositions can be organized with propositional schemes and semantic networks.
\end{definition}

\begin{definition}[Predicate (literacy research analysis)]
Main verbs of clauses or connectives between clauses.
\end{definition}

\begin{definition}[Arguments (literacy research analysis)]
Arguments have functional roles WRT the predicate or can be embedded propositional schemes.
\end{definition}
\end{itemize}

\end{document}

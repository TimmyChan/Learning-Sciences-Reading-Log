
%---------------------------------------%
% Packages arranged by : Tsz Timmy Chan %
%                 Date : May 26th, 2019 % 
%---------------------------------------%

\documentclass{TC}
\usepackage{TCcommon}

\title{TITLE HERE}	% Work Title Here.
\author{Tsz Timmy Chan}	% YOUR NAME HERE 

\usepackage[notes]{TCheader}
\usepackage{TCexamtitle}

\usepackage{setspace}
\linespread{1.5}

%\renewcommand{\benediction}{" " - }
%\renewcommand{\quoteoftheday}{" " \\ - }

\begin{document}
\begin{itemize}
\item Statistical Analysis
	
	\begin{itemize}
		\item Descriptive statistics are not analysis
		\item Parametric VS. Nonparametric
			\begin{itemize}
			\item Parametric analysis to test group \emph{means}; assumes normality, homogeneity of variances, and independence.
			\item Nonparametric analysis to test medians:
			Used when data does not meet the assumptions required for parametric tests - usually when mean and median do not line up.
			\end{itemize}
	\end{itemize}
	\begin{itemize}
	\item Variance Analysis
		\begin{itemize}
		\item t-tests
		\item One-way ANOVA
		\item One way repeated measures ANOVA
		\item Factorial ANOVA
		\item ANCOVA: Analysis of co-variance.
		\item MANOVA: Multiple dependent variables 
		\end{itemize}
	\item Correlational Analysis
		\begin{itemize}
		\item Pearson Correlation---tests for strength of the association of two continuous variables
		\item Spearman Correlation---Tests for the strength of the association between two ordinal variables.
		\item Chi-Squared---Tests for the strength of the association between two categorical variables
		\end{itemize}
	\item Regression Analysis
		\begin{itemize}
		\item Simple Linear Regression---Tests how change in the predictor variable predicts 
		\item Multiple Regression---Linear but high dimensional.
		\end{itemize}
	\end{itemize}
\item Qualitative Coding \& Codebooks	
	\begin{itemize}
	\item Codebook for analysis:
		\begin{itemize}
		\item In qualitative research, a codebook is a set of codes and definitions used as a guide to help when doing analysis
		\item Codes are used to categorize verbatim quations from research participants
		\end{itemize}
	\item Software for qualitative analysis includes ATLAS.ti and NVivo
		\begin{itemize}
		\item Software = more efficient when collaborating
		\item ppl still do this by hand tho cuz it's more "involved"
		\item coding becomes a way to become immersed in the data
		\end{itemize}
	\item Index coding: develop an index of code a priori.
	\item Thematic Coding: create codes that capture core themes that emerge in the research
	\item Grounded Theory: 
		\begin{enumerate}
		\item Open Coding---form initial categories about the phenomenon 
		\item Axial Coding---Assemble data in new ways after open coding 
		\item Selective Coding---identify a storyline and write a story that integrates categories from axial coding
		\end{enumerate}
	\end{itemize}
\item Quantitative Analysis: Deductive (theory driven) and inductive (emerges through data)
	\begin{itemize}
	\item Content analysis---conceptual/relational analysis to understand the actual content and internal features of media
	\item Discourse analysis---linguistic analysis of ongoing flow of communication, looking at verbal and non-verbal communications.
	\item Constant comparison/Grounded Theory---Each interpretation/finding is compared with existing findings as they emerge
	\item Thematic analysis---looks for themes
	\item Narrative Analysis---Focuses on the way people make/use stories to interpret the world around them (structural/functional approach or sociology of stories)
	\item Hermaneutical Analysis---Methodology of interpretation; problems that arise when dealing with meaningful human actions and the products of such actions, most importantly texts.
	\item Phenomenological/Heuristic Analysis===The study of how individuals experience the world. 
	\end{itemize}
\end{itemize}
\end{document}
